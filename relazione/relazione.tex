\documentclass[a4paper,12pt]{report}

\usepackage{alltt, fancyvrb, url}
\usepackage{graphicx}
\usepackage[utf8]{inputenc}
\usepackage{float}
\usepackage{hyperref}
\usepackage[italian]{babel}
\usepackage{appendix}
\usepackage[italian]{cleveref}

\title{Elaborato Basi di Dati}
\author{Desiderio Edoardo}
\begin{document}
\maketitle
\titlepage
\tableofcontents
\newpage

\chapter{Analisi dei requisiti}

\section{intervista}
una	 società	 operante	 nel	 settore	 del	 turismo	 offre	 tra	 i	 suoi	 servizi	 l’organizzazione	 di	 	 visite	
guidate	a	siti	di	interesse	storico-culturale.
Ogni	visita,	opportunamente	descritta,	ha	un	 titolo	 (diverse	visite	hanno	un	 titolo	 ricorrente,	es.	
“Musei	Vaticani	e	Cappella	Sistina”,	“Sito	archeologico	di	Pompei”,	“Galleria	degli	uffizi”,	ecc.),	la	
sua	durata	media		e	il	luogo		in	cui	essa	si	svolge.	Ogni	visita	può	avere	luogo	più	volte	nel	tempo	
secondo	specifici	eventi	programmati.
Gli	eventi,	di	cui	viene	indicato	il	prezzo,	vengono	prenotati	da	gruppi	di	persone	condotti	da	una	
guida	che	illustra	il	percorso	in	una	determinata	lingua;	per	ogni	gruppo	viene	fissata	l’ora	di	inizio	
della	visita	ed	un	numero	minimo	e	massimo	di	partecipanti.
La	società	si	avvale	di	diverse	guide	ognuna	delle	quali	ha	competenze	in	una	o	più	lingue	ad	uno	
specifico	 livello	 di	 conoscenza	 ("B2","C1","C2").	 	 Di	 ogni	 guida	 si	 vuole	
conoscere	 alcuni	 dati	 tra	 i	 quali	 nome,	 sesso,	 data	 di	 nascita,	 titolo	 di	 studio	 e	 relativo	 anno	 di	
conseguimento
I	visitatori,	di	cui	si	vuole	conoscere	almeno	nome,	nazionalità,	lingua	base,	e-mail	e	un	 recapito	
telefonico,	 possono	 aggregarsi	 ad	 uno	 o	 più	 gruppi,	 secondo	 le	 loro	 esigenze.	 Uno	 stesso	
visitatore,	 nel	 tempo,	 può	 partecipare	 a	 gruppi	 diversi	 usando	 ogni	 volta	 una	 certa	 forma	 di	
pagamento	(non	necessariamente	sempre	la	stessa	es.	carta	di	credito,	paypal,	bonifico	bancario)	
della	quale	si	deve	prevedere	la	memorizzazione:	tipologia,	descrizione	e	data	del	pagamento.
Il	 sito	 web	 della	 società	 consente	 la	 visione	 pubblica	 delle	 visite	 organizzate	 e,	 solo	 agli	 utenti	
preventivamente	registrati,	la	prenotazione	di	una	specifica	visita

\section{Rilevamento delle ambiguità e correzioni proposte}
\section{Definizione delle specifiche in linguaggio naturale ed estrazione dei concetti principali}
\newpage
\chapter{Progettazione concettuale}
\section{Schema scheletro}
\section{Raffinamenti proposti}
\section{Schema concettuale finale}
\newpage
\chapter{Progettazione logica}
\section{Stima del volume dei dati}
\section{Descrizione delle operazioni principali e stima della loro frequenza}
\section{Schemi di navigazione e tabelle degli accessi}
\section{Raffinamento dello schema}
\section{Analisi delle ridondanze}
\section{Traduzione di entità e associazioni in relazioni}
\section{Schema relazionale finale}
\section{Traduzione delle operazioni in query SQL}
\newpage
\chapter{Progettazione dell'applicazione}
\end{document}
