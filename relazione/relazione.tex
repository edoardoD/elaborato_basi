\documentclass[a4paper,12pt]{report}

\usepackage{alltt, fancyvrb, url}
\usepackage{graphicx}
\usepackage[utf8]{inputenc}
\usepackage{float}
\usepackage{hyperref}
\usepackage[italian]{babel}
\usepackage{appendix}
\usepackage[italian]{cleveref}
\usepackage{xcolor}
\usepackage{microtype}
\usepackage{array}
\usepackage{adjustbox}
\usepackage{tabularx}

\graphicspath{{./img}}
\title{Elaborato Basi di Dati}
\author{Desiderio Edoardo}
\begin{document}
\maketitle
\titlepage
\tableofcontents
\newpage

\chapter{Analisi dei requisiti}

\section{intervista}
una	 società	 operante	 nel	 settore	 del	 turismo	 offre	 tra	 i	 suoi	 servizi	 l’organizzazione	 di	 	 visite	
guidate	a	siti	di	interesse	storico-culturale.
Ogni	visita,	opportunamente	descritta,	ha	un	 titolo	 (diverse	visite	hanno	un	 titolo	 ricorrente,	es.	
“Musei	Vaticani	e	Cappella	Sistina”,	“Sito	archeologico	di	Pompei”,	“Galleria	degli	uffizi”,	ecc.),	la	
sua	durata	media		e	il	luogo		in	cui	essa	si	svolge.	Ogni	visita	può	avere	luogo	più	volte	nel	tempo	
secondo	specifici	eventi	programmati.
Le escursioni,	di	cui	viene	indicato	il	prezzo,	vengono	prenotati	da	gruppi	di	persone	condotti	da	una	
guida	che	illustra	il	percorso	in	una	determinata	lingua;	per	ogni	gruppo	viene	fissata	l’ora	di	inizio	
della	visita	ed	un	numero	minimo	e	massimo	di	partecipanti.
La	società	si	avvale	di	diverse	guide	ognuna	delle	quali	ha	competenze	in	una	o	più	lingue	ad	uno	
specifico	 livello	 di	 conoscenza	 ("B2","C1","C2"). Di	 ogni	 guida	 si	 vuole	
conoscere	 alcuni	 dati	 tra	 i	 quali	 nome,	 sesso,	 data	 di	 nascita,	 titolo	 di	 studio	 e	 relativo	 anno	 di	
conseguimento
I	visitatori,	di	cui	si	vuole	conoscere	almeno	nome,	nazionalità,	lingua	base,	e-mail	e	un	 recapito	
telefonico,	 possono	 aggregarsi	 ad	 uno	 o	 più	 gruppi,	 secondo	 le	 loro	 esigenze.	 Uno	 stesso	
visitatore,	 nel	 tempo,	 può	 partecipare	 a	 gruppi	 diversi	 usando	 ogni	 volta	 una	 certa	 forma	 di	
pagamento	(non	necessariamente	sempre	la	stessa	es.	carta	di	credito,	paypal,	bonifico	bancario)	
della	quale	si	deve	prevedere	la	memorizzazione:	tipologia,	descrizione	e	data	del	pagamento.
Il	 sito	 web	 della	 società	 consente	 la	 visione	 pubblica	 delle	 visite	 organizzate	 e,	 solo	 agli	 utenti	
preventivamente	registrati,	la	prenotazione	di	una	specifica	visita. In fine l'applicatiov
deve permettere una visione protetta dei dati, quindi non tutti gli utenti ad esempio possono
visionare i gruppi a cui sono affidate le guide 


\section{Rilevamento delle ambiguità e correzioni proposte}
Il testo dell'intervista presenta molte ambiguità. Le principali sono
\begin{itemize}
    \item utilizzo di sinonimi
    \item Elenchi di attributi incompleti
    \item Cartdinalità non specificate
\end{itemize}

Gli attributi parziali e le cardinalità verranno risolti mediante l'uso della logica in fase di creazione dello schema concettuale.
Invece per quanto concerne i sinonimi, è necessario costruire un glossario dei termini



una società operante nel settore del turismo offre tra i suoi servizi l’orga-
nizzazione di visite guidate a siti di interesse storico-culturale.
\textcolor{red}{ Ogni visita,
opportunamente descritta, ha un titolo (diverse visite hanno un titolo ricor-
rente, es. “Musei Vaticani e Cappella Sistina”, “Sito archeologico di Pom-
pei”, “Galleria degli uffizi”, ecc.), la sua durata media e il luogo in cui essa
si svolge. Ogni visita può avere luogo più volte nel tempo secondo specifi-
ci eventi programmati. } 
\textcolor{blue}{Gli eventi, di cui viene indicato il prezzo, vengono
prenotati da gruppi di persone condotti da una guida che illustra il percorso
in una determinata lingua;} 
\textcolor{green}{per ogni gruppo viene fissata l’ora di inizio della
visita ed un numero minimo e massimo di partecipanti} 
\textcolor{orange}{La società si avvale
di diverse guide ognuna delle quali ha competenze in una o più 
lingue ad
uno specifico livello di conoscenza (”B2”,”C1”,”C2”). Di ogni guida si vuole
conoscere alcuni dati tra i quali nome, sesso, data di nascita, titolo di studio
e relativo anno di conseguimento.} 
\textcolor{teal}{ I visitatori, di cui si vuole conoscere almeno
nome, nazionalità, lingua base, e-mail e un recapito telefonico, possono ag-
gregarsi ad uno o più gruppi, secondo le loro esigenze. Uno stesso visitatore,
nel tempo, può partecipare a gruppi diversi usando ogni volta una certa for-
ma di pagamento (non necessariamente sempre la stessa es. carta di credito,
paypal, bonifico bancario) della quale si deve prevedere la memorizzazione:
tipologia, descrizione e data del pagamento} 
\textcolor{cyan}{ Il sito web della società consente
la visione pubblica delle visite organizzate e, solo agli utenti preventivamente
registrati, la prenotazione di una specifica visita} 
% \textcolor{magenta}{domanda} 
% \textcolor{olive}{con} 
% \textcolor{teal}{un}
% \textcolor{violet}{colore} 
% \textcolor{lime}{diverso} 
% \textcolor{pink}{per} \textcolor{gray}{parola}
\subsection*{ipotesi aggiuntive}
dall'intervista fatta si concretizza che:
\begin{itemize}
    \item il dato relativo alla durata media di una visita venga espesso in minuti
    \item  per	uno	specifico	evento	di	visita	guidata	possano	essere	formati	anche	più	gruppi		ognuno	
    col	proprio	orario,	accompagnatore	e	lingua; 
    \item i	 vari	 visitatori	 per	 potersi	 iscriversi	 ad	 uno	 o	 più	 eventi	 debbono	 registrarsi	 sul	 sito	 della	
    società	 fornendo	 e-mail	 e	 recapito	 telefonico.	 La	 banca	 dati	 non	 prevede	 alcuna	 gestione	
    relativamente	 agli	 utenti	 anonimi:	 essi	 possono	 operare	 solo	 per	 funzionalità	 limitate	 di	
    interrogazione	per	vedere	i	dati	degli	eventi	programmati;
    \item per	potersi	iscrivere	ad	un	gruppo	di	visita	relativamente	ad	uno	specifico	evento,	nei	limiti	
    della	 disponibilità	 di	 posti,	 ogni	 visitatore	 registrato	 effettui	 il	 pagamento	 tramite	 carta	 di	
    credito	 (con	 codice	 della	 medesima),	 via	 PayPal	 (l’utente	 deve	 essere	 registrato	 a	 tale	
    servizio),	 o	 tramite	 bonifico	 bancario	 di	 cui	 deve	 fornire	 gli	 estremi	 utilizzando	 il	 campo	
    relativo	alla	descrizione	del	pagamento;
    \item il	prezzo	di	una	visita	sia	comunque	individuale	e	venga	espresso	a	livello	di	evento	in	quanto
    suscettibile	di	variazioni	nel	tempo
    \item per gli utenti con età minore o uguale ai 12 anni è prevista una scontistica del prezzo da un 10 ad un massimo del 20 percento in base ai
    periodi promozionali
\end{itemize}

\subsection*{visite ed eventi}
si possono definire le visite come i siti di interesse che la compagnia di turismo 
ha come partner per trovare collocazione ad eventi diversi. si vuole modellare quindi la 
possibilità di far combaciare più eventi per la stessa visita in tempi e anni diversi
\begin{figure}[H]
    \centering
    \includegraphics[width=0.99\textwidth]{visita-evento.png}
    \caption[short]{una visita può istanziare più eventi metre 
    un evento riferisce per una determinata visita}
    
\end{figure}
\section{Definizione delle specifiche in linguaggio naturale ed estrazione dei concetti principali}
\newpage
\chapter{Progettazione concettuale}
\section{Schema scheletro}
\section{Raffinamenti proposti}
\section{Schema concettuale finale}
\newpage
\chapter{Progettazione logica}
\section{Stima del volume dei dati}
\section{Descrizione delle operazioni principali e stima della loro frequenza}
\section{Schemi di navigazione e tabelle degli accessi}
\section{Raffinamento dello schema}
\section{Analisi delle ridondanze}
\section{Traduzione di entità e associazioni in relazioni}
\section{Schema relazionale finale}
\section{Traduzione delle operazioni in query SQL}
\newpage
\chapter{Progettazione dell'applicazione}
\end{document}
